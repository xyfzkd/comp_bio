\documentclass[answers]{exam}
\usepackage{xeCJK}
\usepackage{zhnumber}
\usepackage{booktabs}
\usepackage{color,framed}
\usepackage{listings}
\usepackage{xcolor}
\usepackage{hyperref}
\pagestyle{headandfoot}
\firstpageheadrule
\firstpageheader{清华大学}{计算生物学}{第一次大作业}
\runningheader{清华大学}
{计算生物学}
{第一次大作业}
\runningheadrule
\firstpagefooter{}{第\thepage\ 页(共\numpages 页)}{}
\runningfooter{}{第\thepage\ 页(共\numpages 页)}{}

\begin{document}
\definecolor{shadecolor}{rgb}{0.92,0.92,0.92}
\lstset{
    numbers=left, 
    numberstyle= \tiny, 
    keywordstyle= \color{ blue!70},
    commentstyle= \color{red!50!green!50!blue!50}, 
    frame=shadowbox, % 阴影效果
    rulesepcolor= \color{ red!20!green!20!blue!20} ,
    escapeinside=``, % 英文分号中可写入中文
    xleftmargin=2em,xrightmargin=2em, aboveskip=1em,
    framexleftmargin=2em
} 
\lstset{ %
backgroundcolor=\color{white},      % choose the background color
basicstyle=\footnotesize\ttfamily,  % size of fonts used for the code
columns=fullflexible,
tabsize=4,
breaklines=true,               % automatic line breaking only at whitespace
captionpos=b,                  % sets the caption-position to bottom
commentstyle=\color{mygreen},  % comment style
escapeinside={\%*}{*)},        % if you want to add LaTeX within your code
keywordstyle=\color{blue},     % keyword style
stringstyle=\color{mymauve}\ttfamily,  % string literal style
frame=single,
rulesepcolor=\color{red!20!green!20!blue!20},
% identifierstyle=\color{red},
% language=c++,
}

\begin{questions}
    \question[10] 设子串$S_1\left[1, \cdots, i\right]$ 和子串$S_2\left[1, \cdots, j\right]$之间的加权编辑距离(weighted edit distance)
    为$D(i,j)$,$i(j)=0,1,2, \cdots, L_{1}\left(L_{2}\right)$。对双序列的全局比对问题,$D(i,j)$的递推关系为:
    
    \begin{equation}
    D(i, j)=\min \left\{\begin{array}{l}{D(i, j-1)+d} \\ {D(i-1, j-1)+p(i, j)} \\ {D(i-1, j)+d}\end{array}\right.
    \end{equation}

    其中,若$S_1(i)=S_2(j)$(精确匹配),则$p(i,j)=0$;若$S_1(i)\neq S_2(j)$(失配),则$p(i,j)=1$;空位罚分$d=2$;且边界条件定义为:$D(0,j)=j\times d$,$D(i,0)=i\times d$。
    
    对附件给出的人类SARS冠状病毒\texttt{BJ01.fasta}和\texttt{TOR2.fasta},试利用Python或MATLAB对上述全局比对算法进行编程,并相应完成:

    \begin{parts}
        \part 以文件形式给出具有最短加权编辑距离的最优全局比对结果;
        \begin{solution}
            最优全局比对结果的文件前16行为:
            %\begin{shaded}
            \begin{lstlisting}
#################################
sequence_id	base_amount
BJ01.fasta	29725
TOR2.fasta	29736
#################################
weighted edit distance	38.0
base matched	29709
base replacement	16
base insertion/deletion	11
#################################
BJ01.fasta:C--C--AGGAAAAGCCAACCAACCTCGATCTCTTGTAGATCTGTTCTCTAAACGAA
    symbol:|  |  ||||||||||||||||||||||||||||||||||||||||||||||||||
TOR2.fasta:CTACCCAGGAAAAGCCAACCAACCTCGATCTCTTGTAGATCTGTTCTCTAAACGAA
CTTTAAAATCTGTGTAGCTGTCGCTCGGCTGCATGCCTAGTGCACCTACGCAGTAT
||||||||||||||||||||||||||||||||||||||||||||||||||||||||
CTTTAAAATCTGTGTAGCTGTCGCTCGGCTGCATGCCTAGTGCACCTACGCAGTAT
                
            \end{lstlisting}
            %\end{shaded}
            更多比对结果详见\href{run:./alignment.txt}{\texttt{alignment.txt}}
        \end{solution}

        \part 以文件形式给出相应的最优编辑转文(edit transcript);
        \begin{solution}
            最优编辑转文的文件前10行为:
            %\begin{shaded}
            \begin{lstlisting}
###########################################
transformation of BJ01.fasta to TOR2.fasta
###########################################
MIIMIIMMMMMMMMMMMMMMMMMMMMMMMMMMMMMMMMMMMMMMMMMMMMMMMMMM
MMMMMMMMMMMMMMMMMMMMMMMMMMMMMMMMMMMMMMMMMMMMMMMMMMMMMMMM
MMMMMMMMMMMMMMMMMMMMMMMMMMMMMMMMMMMMMMMMMMMMMMMMMMMMMMMM
MMMMMMMMMMMMMMMMMMMMMMMMMMMMMMMMMMMMMMMMMMMMMMMMMMMMMMMM
MMMMMMMMMMMMMMMMMMMMMMMMMMMMMMMMMMMMMMMMMMMMMMMMMMMMMMMM
MMMMMMMMMMMMMMMMMMMMMMMMMMMMMMMMMMMMMMMMMMMMMMMMMMMMMMMM
MMMMMMMMMMMMMMMMMMMMMMMMMMMMMMMMMMMMMMMMMMMMMMMMMMMMMMMM
                
            \end{lstlisting}
            %\end{shaded}
            更多比对结果详见\href{run:./transcript.txt}{\texttt{transcript.txt}}
        \end{solution}
        \part 试计算加权编辑距离$D(100,100)$、$D(1000,1000)$和$D(10000,10000)$,并给出总的最短编辑距离$D(L_1,L_2)$。
        \begin{solution}
            \\
            $D\left[100,100\right]=16$ \\
            $D\left[1000,1000\right]=16$ \\
            $D\left[10000,10000\right]=20$ \\
            $D\left[L_1,L_2\right]=38$ \\
        \end{solution}
    \end{parts}
\end{questions}
此次大作业

\end{document}
